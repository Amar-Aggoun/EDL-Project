% !TeX spellcheck = fr_FR

\section{Introduction}
L'Université Constantine 2 propose des diplômes à travers 100 programmes
différents dans plusieurs domaines reconnus dans le monde entier, la raison qui
en fait la destination d'environ 5000 nouveaux étudiants par an, ce qui rend
les démarches administratives très difficiles et conduit parfois à quelques
erreurs, pour cette raison nous proposons un solution informatisée qui va être
plus rapide et plus facile et garantie, par l'automatisation de tous ces
processus de gestion comme le traitement des demandes de transfert,
l'orientation des étudiants etc\ldots \\ Non seulement cela, cette solution est
la meilleure pour minimiser les contacts et les rassemblements répondant aux
protocoles sanitaires vu la situation pandémique dans laquelle nous vivons

\section{Portée De l'application}

\subsection{Objectif de l'application}
Automatiser le traitement des demandes de transfert interne et externe à
l’université selon les critères établis par le ministère de l’enseignement
supérieur, classement des étudiants selon les moyennes obtenus au baccalauréat
, recevoir et traiter les fiches de vœux pour l’orientation de différentes
spécialités de chaque domaine selon les

\subsection{Ressources}
\begin{itemize}
    \item[$\bullet$] Équipe de projet (10 personnes)
    \item[$\bullet$] Équipe d’analyse (3 personnes) meeting chaque weekend-end
    \item[$\bullet$] Équipe de conception (3 personnes) meeting chaque weekend-end
    \item[$\bullet$] Équipe de implémentation (4 personnes) meeting chaque weekend-end
\end{itemize}

\subsection{Livrables}
\begin{itemize}
    \item [$\bullet$]  Application testé et validé qui répond à tous les besoins du client a la fin du
          semestre
\end{itemize}

\subsection{Feuille En route}
\begin{enumerate}
    \item 03/03/2022 Création de l’équipe
    \item 04/03/2022 premier meeting: partition des taches
    \item 06/03/2022 préparation de questionnaire
    \item 07/03/2022 analyse de la description du projet brainstorming en équipe
          se familiarise avec le python et Odoo
\end{enumerate}

\subsection{Vocabulaire Du projet}
\begin{itemize}
    \item[$\bullet$] UC2\@: Université Constantine 2
    \item[$\bullet$] NTIC\@: département de nouvelles technologies de l’information et de la
        communication
    \item[$\bullet$] MI\@: Math et Informatique
    \item[$\bullet$] SCI\@: Sciences de l'information
    \item[$\bullet$] TI\@: Technologies de l'information
    \item[$\bullet$] STIC \@: Sciences et technologie de l’information et communication
    \item[$\bullet$] GL \@:  génie logiciel
    \item[$\bullet$] RSD \@: Réseaux et systèmes distribués
    \item[$\bullet$] SITW \@: Systèmes d'information et technologie Web
\end{itemize}

\subsection{But de document}
\begin{itemize}
    \item[$\bullet$] Identifier les exigences du projet et à faire le point sur les éléments attendus.
    \item[$\bullet$] Établit la liste des objectifs à atteindre et fait le point sur les attentes de manière à affiner le périmètre de votre projet.
    \item[$\bullet$] Donner un cadre au projet.
    \item[$\bullet$] Détermine les exigences du projet dans une optique holistique
    \item[$\bullet$] Donner une documentation plus ou moins détaillé sur le projet au concepteurs
\end{itemize}

\subsection{Références}
Une description du projet des circulaires des différents années fourni par
ministère de l’enseignement supérieur et de la recherche scientifique
\begin{itemize}
    \item[$\bullet$] Livre <<la gestion de projet par étapes analyse des besoins>>
    \item[$\bullet$] Livre <<Les méthodes d’analyse d’enquêtes>>
    \item[$\bullet$] \href{https://asana.com/fr?noredirect=https://www.academia.edu/10419373/LA_GESTION_DE_PROJET_PAR_ÉTAPES_ANAL}{\color{blue}{\underline{Gestion de projet}}}
\end{itemize}